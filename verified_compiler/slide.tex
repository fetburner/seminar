\documentclass[dvipdfmx,cjk,xcolor=dvipsnames,envcountsect,notheorems,aspectratio=169]{beamer}

\usepackage{amsmath,amssymb,amsfonts,amsthm,ascmac,cases,pifont}
\usepackage{pxjahyper}
\usepackage{graphicx}
\usepackage{url}
\usepackage{bussproofs}

\usetheme{sumiilab}

\title{A Verified Compiler for an Impure Functional Language}
\author{水野 雅之}
\date{\today}

\AtBeginSection[]{
\begin{frame}
	\frametitle{Outline}
  \tableofcontents[sectionstyle=show/shaded,subsectionstyle=show/show/hide]
\end{frame}}

\setbeamertemplate{theorems}[numbered]
\theoremstyle{definition}
\newtheorem{definition}{Definition}
\newtheorem{axiom}{Axiom}
\newtheorem{theorem}{Theorem}
\newtheorem{lemma}{Lemma}
\newtheorem{corollary}{Corollary}
\newtheorem{proposition}{Proposition}

\usepackage{listings}
\lstset{
  language={[Objective]Caml},
  basicstyle={\ttfamily\small},
  keywordstyle={\bfseries},
  commentstyle={},
  stringstyle={},
  frame=trlb,
  numbers=none,
  numberstyle={},
  xleftmargin=5pt,
  xrightmargin=5pt,
  keepspaces=true,
  moredelim=[is][\itshape]{@/}{/@},
  moredelim=[is][\color{red}]{@r\{}{\}@},
  moredelim=[is][\color{blue}]{@b\{}{\}@},
  moredelim=[is][\color{DarkGreen}]{@g\{}{\}@},
}

\newcommand{\FIX}{\texttt{fix}}
\newcommand{\LET}{\texttt{let}}
\newcommand{\IN}{\texttt{in}}
\newcommand{\FST}{\texttt{fst}}
\newcommand{\SND}{\texttt{snd}}
\newcommand{\INL}{\texttt{inl}}
\newcommand{\INR}{\texttt{inr}}
\newcommand{\CASE}{\texttt{case}}
\newcommand{\OF}{\texttt{of}}
\newcommand{\REF}{\texttt{ref}}
\newcommand{\RAISE}{\texttt{raise}}
\newcommand{\HANDLE}{\texttt{handle}}
\newcommand{\HALT}{\texttt{halt}}
\newcommand{\FAIL}{\texttt{fail}}
\newcommand{\JMP}{\texttt{jmp}}
\newcommand{\ANS}{\texttt{Ans}}
\newcommand{\EX}{\texttt{Ex}}

\begin{document}
\frame[plain]{\titlepage}

\section*{(My) Motivation}

\begin{frame}
  \frametitle{Motivation}
	\LARGE Planning to verify MinCaml for graduate thesis

	There are several previous reserches (such as CompCert) but...

	\vfill

	MinCaml features
	\begin{itemize}
		\item First class functions
		\item Impure operations
		\item Foreign function interface
	\end{itemize}
	
	\vfill

	Thus tried to survey closely related paper

	"A Verified Compiler for an Impure Functional Language"
\end{frame}

\section*{A Verified Compiler for an Impure Functional Language?}

\begin{frame}
  \frametitle{A Verified Compiler for an Impure Functional Language?}
	\LARGE Adam Chlipala, POPL10

	\vfill

	verifies impure functional language with Coq
	\[
		\begin{array}{lcl}
			e & ::= & c~|~e=e~|~x~|~e~e~|~\FIX~f(x).~e~|~\LET~x=e~\IN~e \\
				&     & |~()~|~\langle e,e\rangle~|~\FST(e)~|~\SND(e)~|~\INL(e)~|~\INR(e) \\
				&     & |~\CASE~e~\OF~\INL(x)\Rightarrow e~|~\INR(x)\Rightarrow e \\
				&     & |~\REF(e)~|~!e~|~e := e~|~\RAISE(e)~|~e~\HANDLE~x\Rightarrow e
		\end{array}
	\]
	$\rightarrow$ close to MinCaml's source language
\end{frame}

\section*{Outline}

\begin{frame}
  \frametitle{Outline}
  \tableofcontents[sectionstyle=show,subsectionstyle=hide]
\end{frame}

\section{Introduction}

\begin{frame}
	\frametitle{Introduction}
	\LARGE A lot of sucks around mechanized proof
	\begin{itemize}
		\item wrong abstractions (e.g. nested variable binders)
		\item copious lemma and case analysis
		\item Once add a new constructor, Modify proof anywhere
	\end{itemize}

	\vfill

	Proposing "engineering" approach to reduce development costs
	\begin{itemize}
		\item {\it Parametric Higher-Order Abstract Syntax}
		\item new semantic approach
		\item sophisticated proof automation
	\end{itemize}
\end{frame}

\begin{frame}
	\frametitle{Source Language}
	\LARGE untyped, subset of ML
	\[
		\begin{array}{lcl}
			e & ::= & c~|~e=e~|~x~|~e~e~|~\FIX~f(x).~e~|~\LET~x=e~\IN~e \\
				&     & |~()~|~\langle e,e\rangle~|~\FST(e)~|~\SND(e)~|~\INL(e)~|~\INR(e) \\
				&     & |~\CASE~e~\OF~\INL(x)\Rightarrow e~|~\INR(x)\Rightarrow e \\
				&     & |~\REF(e)~|~!e~|~e := e~|~\RAISE(e)~|~e~\HANDLE~x\Rightarrow e
		\end{array}
	\]
	no variable-arity features (e.g. sum, product)

	no compound pattern matching
\end{frame}

\begin{frame}
	\frametitle{Target Assembly Language Syntax}
	\LARGE idealized assembly language
	{\Large \[
		\begin{array}{lcl}
			r & ::= & r_0~|~\ldots~|~r_{N-1}\\
			n & \in & \mathbb{N} \\
			L & ::= & r~|~[r+n]~|~[n] \\
			R & ::= & n~|~r~|~[r+n]~|~[n] \\
			I & ::= & L::=R~|~r\mathop{=}\limits^+n~|~L:=R\mathop{=}\limits^?R~|~\texttt{jnz}~R,n \\
			J & ::= & \HALT~R~|~\FAIL~R~|~\JMP~R \\
			B & ::= & (I^*,J) \\
			P & ::= & (B^*,B)
		\end{array}
	\]}
	finite registers and no interface for memory managements
\end{frame}

\begin{frame}
	\frametitle{Source Language Semantics}
	\LARGE big step operational semantics $(h_1, e) \Downarrow (h_2, r)$
	{\footnotesize \begin{prooftree}
		\AxiomC{}
		\UnaryInfC{$(h, \FIX~f(x).~e) \Downarrow (h, \ANS(\FIX~f(x).~e))$}
	\end{prooftree}

	\begin{prooftree}
		\AxiomC{$(h_1, e_1) \Downarrow (h_2, \ANS(\FIX~f(x).~e))$}
		\AxiomC{$(h_2, e_2) \Downarrow (h_3, \ANS(e'))$}
		\AxiomC{$(h_3, e[f \mapsto \FIX~f(x).~e][x \mapsto e']) \Downarrow (h_4, r) $}
		\TrinaryInfC{$(h_1, e_1~e_2) \Downarrow (h_4, r)$}
	\end{prooftree}

	\begin{prooftree}
		\AxiomC{$(h_1, e) \Downarrow (h_2, \EX(v))$}
		\UnaryInfC{$(h_1, e_1~e_2) \Downarrow (h_3, \EX(v))$}
	\end{prooftree}

	\begin{prooftree}
		\AxiomC{$(h_1, e_1) \Downarrow (h_2, \ANS(\FIX~f(x).~e))$}
		\AxiomC{$(h_2, e_2) \Downarrow (h_3, \EX(v))$}
		\BinaryInfC{$(h_1, e_1~e_2) \Downarrow (h_3, \EX(v))$}
	\end{prooftree}

	\begin{prooftree}
		\AxiomC{$(h_1, e) \Downarrow (h_2, \ANS(v))$}
		\UnaryInfC{$(h_1, \REF(e)) \Downarrow (v :: h_2, \ANS(\REF(|h_2|)))$}
	\end{prooftree}

	\begin{prooftree}
		\AxiomC{$(h_1, e) \Downarrow (h_2, \ANS(\REF(n)))$}
		\AxiomC{$h_2.n=v$}
		\BinaryInfC{$(h_1, !e) \Downarrow (h_2, \ANS(v))$}
	\end{prooftree}}
\end{frame}

\begin{frame}
	\frametitle{Source Language Semantics}
	\LARGE {\footnotesize \begin{prooftree}
		\AxiomC{$(h_1, e) \Downarrow (h_2, \ANS(v))$}
		\UnaryInfC{$(h_1, \RAISE(e)) \Downarrow (h_2, \EX(v))$}
	\end{prooftree}

	\begin{prooftree}
		\AxiomC{$(h_1, e_1) \Downarrow (h_2, \EX(v))$}
		\AxiomC{$(h_2, e_2[x \mapsto v]) \Downarrow (h_3, r) $}
		\BinaryInfC{$(h_1, e_1~\HANDLE~x\Rightarrow e_2) \Downarrow (h_3, r)$}
	\end{prooftree}}
	Theorems ignore non-termination.
\end{frame}

\section{Parametric Higher-Order Abstract Syntax}

\begin{frame}
	\frametitle{Parametric Higher-Order Abstract Syntax}
	
\end{frame}

\end{document}

