\documentclass[dvipdfmx,cjk,xcolor=dvipsnames,envcountsect,notheorems,12pt]{beamer}
% * 16:9 のスライドを作るときは、aspectratio=169 を documentclass のオプションに追加する
% * 印刷用の配布資料を作るときは handout を documentclass のオプションに追加する
%   (overlay が全て一つのスライドに出力される)

\usepackage{pxjahyper}% しおりの文字化け対策 (なくても良い)
\usepackage{amsmath,amssymb,amsfonts,amsthm,ascmac,cases,bm,pifont}
\usepackage{graphicx}
\usepackage{url}

% スライドのテーマ
\usetheme{sumiilab}
% ベースになる色を指定できる
%\usecolortheme[named=Magenta]{structure}
% 数式の文字が細くて見難い時は serif の代わりに bold にしましょう
%\mathversion{bold}

%% ===============================================
%% スライドの表紙および PDF に表示される情報
%% ===============================================

%% 発表会の名前とか(省略可)
\session{研究室ゼミ}
%% スライドのタイトル
\title{A Survey of Inlining for Call-by-Need Purely Functional Language}

%% 必要ならば、サブタイトルも
\subtitle{Glasgow Haskell Compiler's Case}
%% 発表者のお名前
\author{水野 雅之}
%% 発表者の所属([] 内は短い名前)
%\institute[東北大学 住井・松田研]{東北大学 工学部 電気情報物理工学科\\住井・松田研究室}% 学部生
\institute[東北大学 住井・松田研]{東北大学 大学院情報科学研究科\\住井・松田研究室}% 院生
%% 発表する日
\date{2016年7月14日}

%% ===============================================
%% 自動挿入される目次ページの設定(削除しても可)
%% ===============================================

%% section の先頭に自動挿入される目次ページ(削除すると、表示されなくなる)
\AtBeginSection[]{
\begin{frame}
  \frametitle{Outline}
  \tableofcontents[sectionstyle=show/shaded,subsectionstyle=show/show/hide]
\end{frame}}
%% subsection の先頭に自動挿入される目次ページ(削除すると、表示されなくなる)
\AtBeginSubsection[]{
\begin{frame}
  \frametitle{アウトライン}
	\tableofcontents[sectionstyle=show/shaded,subsectionstyle=show/shaded/hide]
\end{frame}}

%% 現在の section 以外を非表示にする場合は以下のようにする

%% \AtBeginSection[]{
%% \begin{frame}
%%   \frametitle{アウトライン}
%%   \tableofcontents[sectionstyle=show/hide,subsectionstyle=show/show/hide]
%% \end{frame}}
%% \AtBeginSubsection[]{
%% \begin{frame}
%%   \frametitle{アウトライン}
%%   \tableofcontents[sectionstyle=show/hide,subsectionstyle=show/shaded/hide]
%% \end{frame}}

%% ===============================================
%% 定理環境の設定
%% ===============================================

\setbeamertemplate{theorems}[numbered]% 定理環境に番号を付ける
\theoremstyle{definition}
\newtheorem{definition}{Definition}
\newtheorem{axiom}{Axiom}
\newtheorem{theorem}{Theorem}
\newtheorem{lemma}{Lemma}
\newtheorem{corollary}{Corollary}
\newtheorem{proposition}{Proposition}

%% ===============================================
%% ソースコードの設定
%% ===============================================

\usepackage{listings,jlisting}
%\usepackage[scale=0.9]{DejaVuSansMono}

\definecolor{DarkGreen}{rgb}{0,0.5,0}
% プログラミング言語と表示するフォント等の設定
\lstset{
  language={[Objective]Caml},% プログラミング言語
  basicstyle={\ttfamily\small},% ソースコードのテキストのスタイル
  keywordstyle={\bfseries},% 予約語等のキーワードのスタイル
  commentstyle={},% コメントのスタイル
  stringstyle={},% 文字列のスタイル
  frame=trlb,% ソースコードの枠線の設定 (none だと非表示)
  numbers=none,% 行番号の表示 (left だと左に表示)
  numberstyle={},% 行番号のスタイル
  xleftmargin=5pt,% 左余白
  xrightmargin=5pt,% 右余白
  keepspaces=true,% 空白を表示する
  mathescape,% $ で囲った部分を数式として表示する ($ がソースコード中で使えなくなるので注意)
  % 手動強調表示の設定
  moredelim=[is][\itshape]{@/}{/@},
  moredelim=[is][\color{red}]{@r\{}{\}@},
  moredelim=[is][\color{blue}]{@b\{}{\}@},
  moredelim=[is][\color{DarkGreen}]{@g\{}{\}@},
}

\newcommand{\keyword}[1]{\mathbf{#1}}
\newcommand{\LET}{\keyword{let}}
\newcommand{\REC}{\keyword{rec}}
\newcommand{\ARRAY}{\keyword{Array}}
\newcommand{\CREATE}{\keyword{create}}
\newcommand{\AND}{\keyword{and}}
\newcommand{\IN}{\keyword{in}}
\newcommand{\TRUE}{\keyword{true}}
\newcommand{\WHILE}{\keyword{while}}
\newcommand{\DO}{\keyword{do}}
\newcommand{\DONE}{\keyword{done}}

%% ===============================================
%% 本文
%% ===============================================
\begin{document}
\frame[plain]{\titlepage}% タイトルページ

\section*{Outline}

% 目次を表示させる(section を表示し、subsection は隠す)
\begin{frame}
  \frametitle{Outline}
  \tableofcontents[sectionstyle=show,subsectionstyle=hide]
\end{frame}

\section{Motivation}

\begin{frame}
	\frametitle{Current My Research}
	\large
	Formal verification of MinCaml (esp. K-normalization)
	
	\vfill

	MinCaml : impure higher-order functional language
	\begin{itemize}
		\item Call-by-value
		\item Pair
		\item Array
		\item External function call (possibly cause I/O)
	\end{itemize}

	\vfill

	Good case study for verification of functional programs performing infinite I/O
\end{frame}

\begin{frame}
	\frametitle{Previous Works}
	\large
	Verification of call-by-value languages contain I/O
	\begin{itemize}
		\item Cake ML (ICFP 2016)
		\item Pilsner (ICFP 2015)
	\end{itemize}

	\vfill

	It was red ocean...

	\vfill

	$\Rightarrow$Switch to call-by-need (purely) functional languages
\end{frame}

\section{Call-by-need}

\subsection{What's Call-By-Need}

\begin{frame}
	\frametitle{What's Call-By-Need?}
	\Large
	Improvement of call-by-name

	\begin{center}
		\large
		\begin{tabular}{lll}
			& normalize & duplicate redex \\
			call-by-value & × & × \\
			call-by-name & ○ & ○ \\
			call-by-need & ○ & × \\
		\end{tabular}
	\end{center}
\end{frame}

\begin{frame}
	\frametitle{Call-By-Name}
	\Large
	Yields normal form if exists
	\[\begin{array}{l}
		(\lambda xy.~y)~((\lambda x.~xx)~(\lambda x.~xx)) \\
		\longrightarrow_* \lambda y.~y
	\end{array}\]

	\vfill

	Duplicate redex
	\[\begin{array}{l}
		(\lambda x.~xx)~((\lambda x.~x)~(\lambda x.~x)) \\
		\longrightarrow (\lambda x.~x)~(\lambda x.~x)~((\lambda x.~x)~(\lambda x.~x))
	\end{array}\]
\end{frame}

\begin{frame}
	\frametitle{Call-By-Need}
	\Large
	Yields normal form if exists
	\[\begin{array}{l}
		(\lambda xy.~y)~((\lambda x.~xx)~(\lambda x.~xx)) \\
		\longrightarrow_* \lambda y.~y
	\end{array}\]

	\vfill

	Don't duplicate redex
	\[\begin{array}{l}
		(\lambda x.~xx)~((\lambda x.~x)~(\lambda x.~x)) \\
		\longrightarrow \LET~x=(\lambda x.~x)~(\lambda x.~x)~\IN~x~x \\
		\longrightarrow_* \LET~x=\lambda x.~x~\IN~x~x

	\end{array}\]
	Memoize argument (thunk)
\end{frame}

\subsection{Small-Step Operational Semantic}

\begin{frame}
	\frametitle{Call-By-Name Lambda Calculus}
	\Large
	Syntax
	{\large \[\begin{array}{llcl}
		\texttt{Values} & V & ::= & \lambda x.~M \\
		\texttt{Term} & M, N & ::= & x~|~\lambda x.~M~|~MN\\
		\texttt{Contexts} & E & ::= & []~|~EM
	\end{array}\]}

	\vfill

	Evaluation rule
	{\large \[(\lambda x.~M) N\longrightarrow [x\mapsto N]M\]}
\end{frame}

\begin{frame}
	\frametitle{Introducing Let}
	\Large
	Syntax
	{\normalsize \[\begin{array}{llcl}
		\texttt{Values} & V & ::= & \lambda x.~M \\
		\texttt{Answers} & A & ::= & V~|~\LET~x=M~\IN~A \\
		\texttt{Terms} & L, M, N & ::= & x~|~V~|~MN~|~\LET~x=M~\IN~N\\
	\end{array}\]}

	Evaluation rule
	{\small \[\begin{array}{lcl}
		(\lambda x.~M) N & \longrightarrow & \LET~x=N~\IN~M \\
		\LET~x=V~\IN~C[x] & \longrightarrow & \LET~x=V~\IN~C[V] \\
		(\LET~x=L~\IN~M)N & \longrightarrow & \LET~x=L~\IN~MN \\
		\LET~y=(\LET~x=L~\IN~M)~\IN~N & \longrightarrow & \LET~x=L~\IN~\LET~y=M~\IN~N \\
	\end{array}\]}
\end{frame}

\begin{frame}
	\frametitle{Problem}
	\Large
	Church-Rosser, but nondeterministic
	{\large \[\begin{array}{c}
		(\LET~x=\lambda y.~y~\IN~x)~(\lambda z.~z) \\
		\longrightarrow (\LET~x=\lambda y.~y~\IN~\lambda y.~y)~(\lambda z.~z) \\
	\end{array}\]
	\[\begin{array}{c}
		(\LET~x=\lambda y.~y~\IN~x)~(\lambda z.~z) \\
		\longrightarrow \LET~x=\lambda y.~y~\IN~x~(\lambda z.~z)
	\end{array}\]}
\end{frame}

\begin{frame}
	\frametitle{Standard Reduction}
	\Large
	Syntax
	{\normalsize \[\begin{array}{llcl}
		\texttt{Values} & V & ::= & \lambda x.~M \\
		\texttt{Answers} & A & ::= & V~|~\LET~x=M~\IN~A \\
		\texttt{Terms} & L, M, N & ::= & x~|~V~|~MN~|~\LET~x=M~\IN~N\\
		\texttt{Evaluation} & L, M, N & ::= & []~|~EM~|~\LET~x=M~\IN~E \\
		\texttt{Contexts} & & | & \LET~x=E~\IN~E[x] \\
	\end{array}\]}

	Evaluation rule
	{\small \[\begin{array}{lcl}
		(\lambda x.~M) N & \longrightarrow & \LET~x=N~\IN~M \\
		\LET~x=V~\IN~E[x] & \longrightarrow & \LET~x=V~\IN~E[V] \\
		(\LET~x=M~\IN~A)N & \longrightarrow & \LET~x=M~\IN~AN \\
		\LET~y=(\LET~x=M~\IN~A)~\IN~E[y] & & \\
		\longrightarrow \LET~x=M~\IN~\LET~y=A~\IN~E[y] & & \\
	\end{array}\]}
\end{frame}

\begin{frame}
	\frametitle{Metatheory}
	\renewcommand{\thelemma}{4.2}
	\begin{lemma}
		Given a program $M$, either $M$ is an answer, or there exists a unique evaluation context $E$ and a standard redex $N$ such that $M\equiv E[N]$
	\end{lemma}

	\renewcommand{\thetheorem}{5.8}
	\begin{theorem}[correctness]
		If $M{\mathop{\longrightarrow}\limits_\texttt{name}} _*V$ then, there exists $A$ such that $M{\mathop{\longrightarrow}\limits_\texttt{need}} _*A$
	\end{theorem}
\end{frame}

\begin{frame}
	\frametitle{Let-Less Formulation}
	\Large
	Syntax
	{\normalsize \[\begin{array}{llcl}
		\texttt{Values} & V & ::= & \lambda x.~M \\
		\texttt{Answers} & A & ::= & V~|~(\lambda x.~A) M \\
		\texttt{Terms} & L, M, N & ::= & x~|~V~|~MN \\
		\texttt{Evaluation} & L, M, N & ::= & []~|~EM~|~(\lambda x.~M) E \\
		\texttt{Contexts} & & | & (\lambda x.~E[x])E
	\end{array}\]}

	Evaluation rule
	{\small \[\begin{array}{lcl}
		(\lambda x.~C[x]) V & \longrightarrow & (\lambda x.~C[V]) V \\
		(\lambda x.~L) M N & \longrightarrow & (\lambda x.~LN) M \\
		(\lambda x.~L) ((\lambda y.~M) N) & \longrightarrow & (\lambda y.~(\lambda x.~L)M)N
	\end{array}\]}
\end{frame}

\subsection{Relation to Big-Step Operational Semantics}

\begin{frame}
	\frametitle{Big-Step Operational Semantics}
	\Large
\end{frame}

\end{document}
