\documentclass[dvipdfmx,cjk,xcolor=dvipsnames,envcountsect,notheorems,aspectratio=169]{beamer}

\usepackage{amsmath,amssymb,amsfonts,amsthm,ascmac,cases,bm,pifont}
\usepackage{pxjahyper}
\usepackage{graphicx}
\usepackage{url}
\usepackage{bussproofs}

\usetheme{sumiilab}

\title{The Original Refinement Types}
\subtitle{"Refinement Types for ML"}
\author{水野 雅之}
\date{2015年5月20日}

\AtBeginSection[]{
\begin{frame}
	\frametitle{Outline}
  \tableofcontents[sectionstyle=show/shaded,subsectionstyle=show/show/hide]
\end{frame}}

\setbeamertemplate{theorems}[numbered]
\theoremstyle{definition}
\newtheorem{definition}{Definition}
\newtheorem{axiom}{Axiom}
\newtheorem{theorem}{Theorem}
\newtheorem{lemma}{Lemma}
\newtheorem{corollary}{Corollary}
\newtheorem{proposition}{Proposition}

\usepackage{listings,jlisting}
\lstset{
  language={[Objective]Caml},
  basicstyle={\ttfamily\small},
  keywordstyle={\bfseries},
  commentstyle={},
  stringstyle={},
  frame=trlb,
  numbers=none,
  numberstyle={},
  xleftmargin=5pt,
  xrightmargin=5pt,
  keepspaces=true,
  moredelim=[is][\itshape]{@/}{/@},
  moredelim=[is][\color{red}]{@r\{}{\}@},
  moredelim=[is][\color{blue}]{@b\{}{\}@},
  moredelim=[is][\color{DarkGreen}]{@g\{}{\}@},
}

\newcommand{\BOOL}{\texttt{Bool}}
\newcommand{\TRUE}{\texttt{true}}
\newcommand{\FALSE}{\texttt{false}}
\newcommand{\IF}{\texttt{if}}
\newcommand{\THEN}{\texttt{then}}
\newcommand{\ELSE}{\texttt{else}}
\newcommand{\CONS}{\texttt{cons}}
\newcommand{\NIL}{\texttt{nil}}
\newcommand{\SINGLETON}{\texttt{singleton}}
\newcommand{\LIST}{\texttt{list}}
\newcommand{\RECTYPE}{\texttt{rectype}}
\newcommand{\AND}{\texttt{and}}

\begin{document}
\frame[plain]{\titlepage}

\section*{Outline}

\begin{frame}
  \frametitle{Outline}
  \tableofcontents[sectionstyle=show,subsectionstyle=hide]
\end{frame}

\section{Motivation}

\begin{frame}
  \frametitle{Motivation}
	\Large Nowadays refinement types are popular
	\[\Lambda \pi:\tau.~e\qquad\{x : T~|~P(x)\}\]
	\begin{itemize}
		\item Isn't it dependent types?
		\item Too many extensions to see through the essence
	\end{itemize}
	\vfill
	Try to read original paper ("Refinement Types for ML")
\end{frame}

\section{Introduction}

\begin{frame}
	\frametitle{Refinement Types?}
	\Large "We describe a refinement of ML's type system
	allowing the specification of recursively defined subtypes of user-defined datatypes."
	\begin{itemize}
		\item Preserve desirable properties of ML's type system
			\begin{itemize}
				\item Decidability of type inference
				\item Every well-typed expression has a principal type
			\end{itemize}
		\item Allow more errors to be detected at compile-time
	\end{itemize}
\end{frame}

\begin{frame}[fragile]
	\frametitle{Opportunity to Improve}
\begin{lstlisting}
datatype 'a list = nil | cons of 'a * 'a list
fun lastcons (last as cons(hd,nil)) = last
  | lastcons (cons(hd, tl)) = lastcons tl
\end{lstlisting}
	\begin{itemize}
		\item undefined when called on nil
	\end{itemize}
\begin{lstlisting}
case lastcons y of
     cons(x,nil) => print x
\end{lstlisting}
	\begin{itemize}
		\item ML type system does not distinguish singleton lists
		\item Get compiler warning
	\end{itemize}
\end{frame}

\begin{frame}[fragile]
	\frametitle{Solution}
\begin{lstlisting}
datatype 'a list = nil | cons of 'a * 'a list
rectype 'a singleton = cons('a, nil)
\end{lstlisting}
	\[
		\begin{array}{rlclc}
			\CONS : &(\alpha * \alpha~?\NIL) & \to & \alpha~\SINGLETON & \land \\
							&(\alpha * \alpha~\SINGLETON) & \to & \alpha~\LIST & \land \\
							&(\alpha * \alpha~\LIST) & \to & \alpha~\LIST &
		\end{array}
	\]
	\begin{itemize}
		\item \tt{rectype} declaration stand for subtypes
		\item $\alpha~\SINGLETON=\{\CONS(\alpha, \NIL)~|~a\in \alpha\}\subset \alpha~\LIST$
		\item Abstract interpretation over lattice (p. 2)
		\item Intersection type (only refinement types)
	\end{itemize}
\end{frame}

\begin{frame}
	\frametitle{Expressivity}
	\Large There are examples which cannot be specified as refinement types
	\begin{itemize}
		\item List without repeated elements
		\item Closed term in $\lambda$-calculus
	\end{itemize}
	\tt{rectype} specify so-called regular tree sets
\end{frame}

\section{The Language}

\begin{frame}
	\frametitle{Rectype Declarations}
	\[
		\begin{array}{lcl}
			\textit{rectype} & ::= & \RECTYPE~\textit{rectypedecl} \\
			\textit{rectypedecl} & ::= & <\textit{\textit{mlty}pevar}>~\textit{reftyname} = \textit{recursivety}~<\AND~\textit{rectypedecl}> \\
			\textit{recursivety} & ::= & \textit{recursivety}~|~\textit{recursivety} \\
									& & \textit{mlty} \to \textit{recursivety} \\
									& & \textit{constructor}~\textit{recursivetyseq} \\
									& & \textit{mltypevar} \\
									& & <\textit{mltypevar}>~\textit{reftyname}
		\end{array}
	\]
	\Large Each rectype declaration must be consistent with the ML datatype
	\begin{itemize}
		\item e.g. $\RECTYPE~\alpha~\texttt{bad} = \NIL (\NIL)$
	\end{itemize}
	Recursive type's definition must have the same type variable argument
\end{frame}

\begin{frame}
	\frametitle{Refinement Types}
	\[
		\begin{array}{lcl}
			\textit{refty} & ::= & \textit{refty} \land \textit{refty} \\
						& & \textit{refty} \lor \textit{refty} \\
						& & \textit{refty} \to \textit{refty} \\
						& & \bot \\
						& & <\textit{refty}> \textit{mltyname} \\
						& & <\textit{refty}> \textit{reftyname} \\
						& & <\textit{refty}> \textit{reftyname} \\
						& & \textit{reftyvar} :: \textit{mltyvar}
		\end{array}
	\]
	\Large Refinement type variable is bounded by an ML type variable \\
	Ranges only over the refinements of an ML type
\end{frame}

\begin{frame}[fragile]
	\frametitle{An Example}
	Representation of natural numbers in binary
\begin{lstlisting}
datatype bitstr =
   e | z of bitstr | o of bitstr
\end{lstlisting}
	We would like to guarantee that zero does not appear in MSB (standard form, \tt{std})
\begin{lstlisting}
rectype std = e | stdpos
and stdpos = o(e) | z(stdpos) | o(stdpos)
\end{lstlisting}
\begin{lstlisting}
fun add e m = m
  | add n e = n
  | add (z n) (z m) = z (add n m)
  | add (o n) (z m) = o (add n m)
  | add (z n) (o m) = o (add n m)
  | add (o n) (o m) = z (add (add (o e) n) m)
\end{lstlisting}
	Infered type (See p. 4)
\end{frame}

\section{From Rectype Declarations to Datatype Lattices}

\begin{frame}[fragile]
	\frametitle{From Rectype Declarations to Datatype Lattices}
\begin{lstlisting}
datatype bitstr =
    e | z of bitstr | o of bitstr
rectype std = e | stdpos
and stdpos = o(e) | z(stdpos) | o(stdpos)
\end{lstlisting}
	{\Large Manipulating regular tree grammers, we can infer:}
	\begin{itemize}
		\item These refinement types are closed under intersection and union (See p. 5)
		\item They form lattice
	\end{itemize}
	{\Large Types for the constructors are calculated as}
	\frametitle{}
	\[
		\begin{array}{lcl}
			\texttt{e} & : & ?\texttt{e} \\
			\texttt{o} & : & ?\texttt{e} \to \texttt{stdpos} \land \texttt{stdpos} \to \texttt{stdpos} \\
			\texttt{z} & : & \texttt{stdpos} \to \texttt{stdpos}
		\end{array}
	\]
\end{frame}

\section{From Datatype Lattice to Function Types}

\begin{frame}
	\frametitle{Normal Form}
	{\Large After apply the following rewrite rules,}
	\[
		\begin{array}{lcl}
			\rho \land (\sigma \lor \tau) & \Rightarrow & (\rho \land \sigma) \lor (\rho \land \tau) \\
			(\rho \lor \sigma) \to \tau & \Rightarrow & (\rho \to \tau) \land (\sigma \to \tau)
		\end{array}
	\]
	{\Large The refinement types will fit the grammar}
	\[
		\begin{array}{lcl}
			\textit{unf} & ::= & \textit{inf} \\
					& & \textit{unf} \lor \textit{unf} \\
			\textit{inf} & ::= & <\textit{unf}>~\textit{reftyname} \\
					& & \textit{inf} \land \textit{inf} \\
					& & \textit{inf} \to \textit{unf} \\
					& & \textit{reftyvar} :: \textit{mltyvar}
		\end{array}
	\]
\end{frame}

\begin{frame}
	\frametitle{Subtyping Rule for Arrow}
	{\Large "$\to$" is contracariant in its first argument and covariant in its second argument}
	\begin{prooftree}
		\AxiomC{$\tau_1 \leq \tau_2$}
		\AxiomC{$\sigma_2 \leq \sigma_1$}
		\RightLabel{\textsc{S-Arrow}}
		\BinaryInfC{$\sigma_1 \to \tau_1 \leq \sigma_2 \to \tau_2$}
	\end{prooftree}
	{\Large Datatype constructor may also be covariant or contracariant in their arguments}
	\begin{itemize}
		\item $\texttt{stdpos}~\texttt{list} \leq \texttt{std}~\texttt{list}$
	\end{itemize}
\end{frame}

\begin{frame}
	\frametitle{Subtyping Rule for Union}
	\begin{prooftree}
		\AxiomC{If for each $\sigma_i$ there is a ${\sigma'}_j$ such that $\sigma_i \leq {\sigma'}_j$}
		\RightLabel{\textsc{S-Union}}
		\UnaryInfC{$\sigma_1 \lor \sigma_2 \lor \ldots \lor \sigma_n \leq {\sigma'}_1 \lor {\sigma'}_2 \leq \ldots \lor {\sigma'}_m$}
	\end{prooftree}
	{\Large $\sigma_i$, ${\sigma'}_j$ : inf refinement types}
\end{frame}

\begin{frame}
	\frametitle{Refinement Type of Application}
	{\Large Function has type $\sigma = (\rho_1\to\tau_1) \land (\rho_2\to\tau_2) \land \ldots \land (\rho_n\to\tau_n)$ \\
	Argument has type $\rho$\\
	Their application $\texttt{apptype}(\sigma, \rho$)}
	\[\texttt{apptype}(\sigma, \rho)=\bigwedge_{\{i|\rho\leq\rho_i\}}\tau_i\]
\end{frame}

\section{Polymorphism}

\begin{frame}
	\frametitle{Polymorphism}
	{\Large The domain of type variable is restricted to range over subtypes of given bound (bounded quantification)}
	\[
		\begin{array}{lcl}
			\textit{reftyscheme} & ::= & \textit{inf} \\
													 & & \forall \alpha .~\textit{reftyscheme} \\
													 & & \forall r\alpha :: \alpha.~\textit{reftyscheme}
		\end{array}
	\]
	{\Large Refinement type for identity function id:}
	\[
		\forall \alpha. \forall r\alpha :: \alpha.~r\alpha \to r\alpha
	\]
	{\Large Instantiate $\alpha$ to bitstr and instantiate the refinement type quantifier}
	\[
		\begin{array}{cccl}
			\texttt{bitstr} & \to & \texttt{bitstr} & \land \\
			\texttt{stdpos} & \to & \texttt{stdpos} & \land \\
			\texttt{std} & \to & \texttt{std} & \land \\
			?\texttt{e} & \to & ?\texttt{e} & \land \\
			\bot & \to & \bot & 
		\end{array}
	\]
\end{frame}

\section{The Type Inference Algorithm}

\begin{frame}
	\frametitle{Target Language}
	\[
		\begin{array}{lcl}
			\textit{exp} & ::= & \textit{variable} \\
							& & \textit{exp}~\textit{exp} \\
							& & \lambda~\textit{variable}.~\textit{exp} \\
							& & \textit{exp} : \textit{refty} \\
							& & \texttt{let}~\textit{variable} = \textit{exp}~\texttt{in}~\textit{exp} \\
							& & \texttt{fix}~\textit{variable}. \textit{exp}
		\end{array}
	\]
	{\Large
		\begin{tabular}{ll}
			Typing derivation & $\Gamma \vdash e : D :: L$ \\
			Meaning of metavariables & See p. 7
		\end{tabular}}
\end{frame}

\begin{frame}
	\frametitle{The Type Inference Algorithm}
	{\Large type inference by unification just like ML types \\
	Typing rules : See Figure 2\\
	Eliminate all of the refinement, corresponds to ML types}
	\begin{theorem}[preservation]
		For all valid type environments $\Gamma$ and expressions e, if e evaluates to v and $\Gamma \vdash e : D :: L$ then $\Gamma \vdash v : D' :: L$ for some $D' \leq D$.
	\end{theorem}
	\begin{proof}
		By induction on the structure of the definition of the "evaluates to" relation
	\end{proof}
\end{frame}

\begin{appendix}
	\begin{frame}[fragile]
		\frametitle{Manipulating Case Statement}
		\Large Implicitly define a new constant \tt{CASE}\_\it{datatype}
\begin{lstlisting}
case E of
    e => E1
  | o(m) => E2
  | z(m) => E3
\end{lstlisting}
\begin{lstlisting}
CASE_bitstr E
  (fn () => E1)
  (fn (m) => E2)
  (fn (m) => E3)
\end{lstlisting}
		Type for \tt{CASE\_bitstr}: See Figure 1
	\end{frame}
\end{appendix}

\end{document}
