\documentclass[dvipdfmx,cjk,xcolor=dvipsnames,envcountsect,notheorems,12pt]{beamer}
% * 16:9 のスライドを作るときは、aspectratio=169 を documentclass のオプションに追加する
% * 印刷用の配布資料を作るときは handout を documentclass のオプションに追加する
%   (overlay が全て一つのスライドに出力される)

\usepackage{pxjahyper}% しおりの文字化け対策 (なくても良い)
\usepackage{amsmath,amssymb,amsfonts,amsthm,ascmac,cases,bm,pifont}
\usepackage{graphicx}
\usepackage{url}
\usepackage{bussproofs}

% スライドのテーマ
\usetheme{sumiilab}
% ベースになる色を指定できる
%\usecolortheme[named=Magenta]{structure}
% 数式の文字が細くて見難い時は serif の代わりに bold にしましょう
%\mathversion{bold}

%% ===============================================
%% スライドの表紙および PDF に表示される情報
%% ===============================================

%% 発表会の名前とか(省略可)
\session{研究室ゼミ}
%% スライドのタイトル
\title{A Natural Semantics for \mbox{Lazy Evaluation}}
%% 必要ならば、サブタイトルも
%\subtitle{}
%% 発表者のお名前
\author{水野雅之}
%% 発表者の所属([] 内は短い名前)
%\institute[東北大学 住井・松田研]{工学部 電気情報物理工学科\\住井・松田研究室}% 学部生
%\institute[東北大学 住井・松田研]{大学院情報科学研究科 情報基礎科学専攻\\住井・松田研究室}% 院生
%% 発表する日
\date{2016年11月30日}

%% ===============================================
%% 自動挿入される目次ページの設定(削除しても可)
%% ===============================================

%% section の先頭に自動挿入される目次ページ(削除すると、表示されなくなる)
\AtBeginSection[]{
\begin{frame}
  \frametitle{アウトライン}
  \tableofcontents[sectionstyle=show/shaded,subsectionstyle=show/show/hide]
\end{frame}}
%% subsection の先頭に自動挿入される目次ページ(削除すると、表示されなくなる)
%\AtBeginSubsection[]{
%\begin{frame}
%  \frametitle{アウトライン}
%  \tableofcontents[sectionstyle=show/shaded,subsectionstyle=show/shaded/hide]
%\end{frame}}

%% 現在の section 以外を非表示にする場合は以下のようにする

%% \AtBeginSection[]{
%% \begin{frame}
%%   \frametitle{アウトライン}
%%   \tableofcontents[sectionstyle=show/hide,subsectionstyle=show/show/hide]
%% \end{frame}}
%% \AtBeginSubsection[]{
%% \begin{frame}
%%   \frametitle{アウトライン}
%%   \tableofcontents[sectionstyle=show/hide,subsectionstyle=show/shaded/hide]
%% \end{frame}}

%% ===============================================
%% 定理環境の設定
%% ===============================================

\setbeamertemplate{theorems}[numbered]% 定理環境に番号を付ける
\theoremstyle{definition}
\newtheorem{definition}{定義}
\newtheorem{axiom}{公理}
\newtheorem{theorem}{定理}
\newtheorem{lemma}{補題}
\newtheorem{corollary}{系}
\newtheorem{proposition}{命題}

%% ===============================================
%% ソースコードの設定
%% ===============================================

\usepackage{listings,jlisting}
%\usepackage[scale=0.9]{DejaVuSansMono}

\definecolor{DarkGreen}{rgb}{0,0.5,0}
% プログラミング言語と表示するフォント等の設定
\lstset{
  language={[Objective]Caml},% プログラミング言語
  basicstyle={\ttfamily\small},% ソースコードのテキストのスタイル
  keywordstyle={\bfseries},% 予約語等のキーワードのスタイル
  commentstyle={},% コメントのスタイル
  stringstyle={},% 文字列のスタイル
  frame=trlb,% ソースコードの枠線の設定 (none だと非表示)
  numbers=none,% 行番号の表示 (left だと左に表示)
  numberstyle={},% 行番号のスタイル
  xleftmargin=5pt,% 左余白
  xrightmargin=5pt,% 右余白
  keepspaces=true,% 空白を表示する
  mathescape,% $ で囲った部分を数式として表示する ($ がソースコード中で使えなくなるので注意)
  % 手動強調表示の設定
  moredelim=[is][\itshape]{@/}{/@},
  moredelim=[is][\color{red}]{@r\{}{\}@},
  moredelim=[is][\color{blue}]{@b\{}{\}@},
  moredelim=[is][\color{DarkGreen}]{@g\{}{\}@},
}

\newcommand{\keyword}[1]{\mathbf{#1}}
\newcommand{\LET}{\keyword{let}}
\newcommand{\REC}{\keyword{rec}}
\newcommand{\ARRAY}{\keyword{Array}}
\newcommand{\CREATE}{\keyword{create}}
\newcommand{\AND}{\keyword{and}}
\newcommand{\IN}{\keyword{in}}
\newcommand{\TRUE}{\keyword{true}}
\newcommand{\WHILE}{\keyword{while}}
\newcommand{\DO}{\keyword{do}}
\newcommand{\DONE}{\keyword{done}}
\newcommand{\BVAR}{\keyword{bvar}}
\newcommand{\FVAR}{\keyword{fvar}}
\newcommand{\ABS}{\keyword{abs}}
\newcommand{\APP}{\keyword{app}}

%% ===============================================
%% 本文
%% ===============================================
\begin{document}
\frame[plain]{\titlepage}% タイトルページ

\section*{アウトライン}

% 目次を表示させる(section を表示し、subsection は隠す)
%\begin{frame}
%  \frametitle{アウトライン}
%  \tableofcontents[sectionstyle=show,subsectionstyle=hide]
%\end{frame}

\begin{frame}
	\frametitle{これまでの研究}
	\begin{center}
		\Large 入出力を含む純粋でない関数型言語の\mbox{K正規化}を検証
		\vfill
		$\Downarrow$
		\vfill
		CakeML(ICFP2016)やPilsner(ICFP2015)\\
		$\rightarrow$call by valueな関数型言語処理系の検証はred oceanっぽい
		\vfill
		$\Downarrow$
		\vfill
		call by need(以後cbn)な関数型言語へ
	\end{center}
\end{frame}

\begin{frame}
	\frametitle{今回紹介する論文}
	\begin{itemize}
		\item A Natural Semantics for Lazy Evaluation
		\item A locally nameless representation for natural semantics for lazy evaluation
			\begin{itemize}
				\item $\uparrow\uparrow$にlocally nameless representationを導入したもの
			\end{itemize}
	\end{itemize}
\end{frame}

\section{A Natural Semantics for Lazy Evaluation}

\begin{frame}
	\frametitle{A Natural Semantics for Lazy Evaluation}
	\begin{itemize}
		\item cbnな$\lambda$計算の拡張の大ステップ操作的意味論を定義する論文
			\begin{itemize}
				\item $\lambda$計算に相互再帰的なletを入れた言語が対象
				\item 変数名をthunkのlocationとして流用
					\begin{itemize}
						\item 時々束縛変数のrenamingが必要
					\end{itemize}
			\end{itemize}
	\end{itemize}
\end{frame}

\begin{frame}
	\frametitle{論文のモチベーション}
	\begin{itemize}
		\item 実際の処理系の実装を模した意味論を\mbox{定義}したい
			\begin{itemize}
				\item 値のsharing
				\item 一度評価された関数引数は再び評価されない
			\end{itemize}
		\item 先行研究はtoo concrete
			\begin{itemize}
				\item 継続、コードポインタ、スタックポインタ、間接参照を含むノード等
				\item 証明に使うには大変
			\end{itemize}
	\end{itemize}
\end{frame}

\begin{frame}
	\frametitle{対象言語}
	\begin{itemize}
		\item $\lambda$計算に相互再帰なletを導入した言語
		\item 評価に先立ち構文を制限
	\end{itemize}
	\Large
	\[\begin{array}{lcl}
		e & ::= & \lambda x.~e \\
			& | & e~x \\
			& | & x \\
			& | & \LET~x_1=e_1,\cdots,x_n=e_n~\IN~e
	\end{array}\]
\end{frame}

\begin{frame}
	\frametitle{評価規則}
	\Large
	\fbox{$\Gamma : e \Downarrow \Delta : z$}
	\begin{prooftree}
		\AxiomC{}
		\UnaryInfC{$\Gamma : \lambda x.e \Downarrow \Gamma : \lambda x.e$}
	\end{prooftree}
	\vfill
	\begin{prooftree}
		\AxiomC{$\Gamma : e \Downarrow \Delta : \lambda y.e'$}
		\AxiomC{$\Delta : e'[x/y] \Downarrow \Theta : z$}
		\BinaryInfC{$\Gamma : e~x \Downarrow \Theta : z$}
	\end{prooftree}
	\vfill
	\begin{prooftree}
		\AxiomC{$\Gamma : e \Downarrow \Delta : z$}
		\UnaryInfC{$(\Gamma,x\mapsto e) : x \Downarrow (\Delta,x\mapsto z) : z$}
	\end{prooftree}
	\vfill
	\begin{prooftree}
		\AxiomC{$(\Gamma,x_1\mapsto e_1 \cdots x_n\mapsto e_n) : e \Downarrow \Delta : z$}
		\UnaryInfC{$\Gamma : \LET~x_1=e_1, \cdots ,x_n=e_n~\IN~x \Downarrow \Delta : z$}
	\end{prooftree}
\end{frame}

\begin{frame}
	\frametitle{評価の例}
	\LARGE \begin{center}
		必要なら白板で…
		\vfill
	\end{center}

	\begin{itemize}
		\item 評価関係は帰納的に定義するため、無限回の規則適用は許さない
		\item 一見無限回評価規則を適用できそうな式の評価でも、有限回の適用で終了する場合がある
		\item 評価の途中でrenamingが必要になる場合がある
	\end{itemize}
\end{frame}

\begin{frame}
	\frametitle{表示的意味論との対応}
	\begin{center}
		Coqでの検証に使えるか微妙なので、あまり詳しく追ってません
	\end{center}
	\begin{itemize}
		\item 最小不動点を使う、ありがちな\mbox{表示的意味論}
		\item Correctnessはいわゆる健全性
		\item Computational adequacyはいわゆる健全性+完全性
	\end{itemize}
\end{frame}

\begin{frame}
	\frametitle{意味論の拡張}
	\begin{itemize}
		\item コンストラクタの追加
			\begin{itemize}
				\item コンストラクタの中身は変数に制限
				\item コンストラクタも値とする
			\end{itemize}
		\item GCの実装
			\begin{itemize}
				\item 環境がヒープに相当
				\item 全体の式から到達できない変数の束縛を削除
			\end{itemize}
		\item 評価のコストのカウント
			\begin{itemize}
				\item 計算に要したステップ数の評価ができる
			\end{itemize}
	\end{itemize}
\end{frame}

\section{A locally nameless representation for natural semantics for lazy evaluation}

\begin{frame}
	\frametitle{A locally nameless representation for natural semantics for lazy evaluation}
	\begin{itemize}
		\item 先に解説した論文にlocally nameless representationを導入する論文
		\item Correctnessやcomputational adequacyは証明していない
		\item (この時点では)Coqを使っていない
			\begin{itemize}
				\item 後にCoqで書いた論文も出ている模様
			\end{itemize}
	\end{itemize}
\end{frame}

\begin{frame}
	\frametitle{論文のモチベーション}
	\begin{itemize}
		\item 先に挙げた論文の意味論は変数名に深く依存している
			\begin{itemize}
				\item 評価前にrenamingが必要
				\item 評価中に必要になる場合も
			\end{itemize}
		\item 名前で束縛を表現するため定理証明支援系での扱いが煩雑
	\end{itemize}
	\vfill
	\LARGE
	\begin{center}$\downarrow$\end{center}
	\vfill
	\begin{center}これらを解決したい\end{center}
\end{frame}

\begin{frame}
	\frametitle{Locally nameless representation}
	\begin{itemize}
		\item 自由変数を名前で、束縛変数を\mbox{de Bruijn index}で表現する手法
			\begin{itemize}
				\item シフト``は''不要になる
			\end{itemize}
	\end{itemize}
	\large 

	\begin{eqnarray*}
		v & ::= & \BVAR~i~j \\
			& | & \FVAR~x \\
		t & ::= & v \\
			& | & \ABS~t \\
			& | & \APP~t~v \\
			& | & \LET~\{t_i\}^n_{i=1}~\IN~t
	\end{eqnarray*}
\end{frame}

\begin{frame}
	\frametitle{Variable opening}
	\Large 
	束縛の中身を処理する場合、束縛されていた変数が自由変数になる\\
	$\rightarrow$インデックスで表現された変数を名前による表現に直す
	\vfill
	{\footnotesize \begin{eqnarray*}
		\{k\rightarrow \overline{x}\}(\BVAR~i~j) & = &
			\begin{cases}
				\FVAR~(\keyword{List.nth}~j~\overline{x} & \texttt{if}~i=k \land j < |\overline{x}| \\
				\BVAR~i~j & \texttt{otherwise} \\
			\end{cases} \\
		\{k\rightarrow \overline{x}\}(\FVAR~x) & = & \FVAR~x \\
		\{k\rightarrow \overline{x}\}(\ABS~t) & = & \ABS~(\{k + 1\rightarrow \overline{x}\}~t) \\
		\{k\rightarrow \overline{x}\}(\APP~t~v) & = & \APP~(\{k\rightarrow \overline{x}\}~t)~(\{k\rightarrow \overline{x}\}~v) \\
		\{k\rightarrow \overline{x}\}(\LET~\overline{t}~\IN~t) & = & \LET~(\{k+1\rightarrow \overline{x}\}\overline{t}~\IN~(\{k+1\rightarrow \overline{x}\}~t)
	\end{eqnarray*}}
\end{frame}

\begin{frame}
	\frametitle{Variable closing}
	\Large 
	変数を束縛する際、束縛された変数をインデックスでの表現に直す\\
	$\rightarrow$Openingの逆の操作
	\vfill
	{\footnotesize \begin{eqnarray*}
		\{k\leftarrow \overline{x}\}(\BVAR~i~j) & = & \BVAR~i~j \\
		\{k\leftarrow \overline{x}\}(\FVAR~x) & = & 
			\begin{cases}
				\BVAR~k~j & \texttt{if}~\exists j:0\leq j < |\overline{x}|.x=\keyword{List.nth}~j~\overline{x} \\
				\FVAR~x & \texttt{otherwise} \\
			\end{cases} \\
		\{k\leftarrow \overline{x}\}(\ABS~t) & = & \ABS~(\{k + 1\leftarrow \overline{x}\}~t) \\
		\{k\leftarrow \overline{x}\}(\APP~t~v) & = & \APP~(\{k\leftarrow \overline{x}\}~t)~(\{k\leftarrow \overline{x}\}~v) \\
		\{k\leftarrow \overline{x}\}(\LET~\overline{t}~\IN~t) & = & \LET~(\{k+1\leftarrow \overline{x}\}\overline{t}~\IN~(\{k+1\leftarrow \overline{x}\}~t)
	\end{eqnarray*}}
\end{frame}

\begin{frame}
	\frametitle{Local closure}
	\Large
	\begin{itemize}
		\item 自由変数は名前で、束縛変数はインデックスで正しく表現されているかを表す述語
	\end{itemize}
\end{frame}

\end{document}
